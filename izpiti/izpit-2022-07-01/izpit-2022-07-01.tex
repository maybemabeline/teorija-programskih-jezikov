\documentclass[arhiv]{../izpit}
\usepackage{amssymb}
\usepackage{fouriernc}
\usepackage{mathpartir}

\begin{document}

\newcommand{\bnfis}{\mathrel{{:}{:}{=}}}
\newcommand{\bnfor}{\;\mid\;}
\newcommand{\fun}[2]{\lambda #1. #2}
\newcommand{\conditional}[3]{\mathtt{if}\;#1\;\mathtt{then}\;#2\;\mathtt{else}\;#3}
\newcommand{\whileloop}[2]{\mathtt{while}\;#1\;\mathtt{do}\;#2}
\newcommand{\recfun}[3]{\mathtt{rec}\;#1\;#2. #3}
\newcommand{\funty}[2]{#1 \to #2}
\newcommand{\tru}{\mathtt{true}}
\newcommand{\fls}{\mathtt{false}}
\newcommand{\tbool}{\mathtt{bool}}
\newcommand{\intsym}[1]{\underline{#1}}
\newcommand{\kwd}[1]{\mathbf{#1}}
\newcommand{\kwdpre}[1]{\kwd{#1}\;}
\newcommand{\kwdmid}[1]{\;\kwd{#1}\;}
\newcommand{\kwdpost}[1]{\;\kwd{#1}}
\newcommand{\true}{\kwd{true}}
\newcommand{\false}{\kwd{false}}
\newcommand{\cond}[3]{\kwdpre{if} #1 \kwdmid{then} #2 \kwdmid{else} #3}
\newcommand{\whiledocmd}[2]{\kwdpre{while} #1 \kwdmid{do} #2}
\newcommand{\skipcmd}{\kwd{skip}}
\newcommand{\ttt}{\!t}
\newcommand{\fff}{\!\!f}
% \newcommand{\infer}[2]{\displaystyle{\frac{#1}{#2}}}
\newcommand{\boolty}{\kwd{bool}}
\newcommand{\intty}{\kwd{int}}
\newcommand{\eqs}{\mathcal{E}}
\newcommand{\ctxt}{\mathcal{C}}
\newcommand{\itp}[1]{[\![ #1 ]\!]}
\newcommand{\return}{\kwdpre{return}}
\newcommand{\letin}[1]{\kwdpre{let} #1 \kwdmid{in}}
\newcommand{\bind}{\mathop{>\!\!\!>\!\!\!=}}

\makeatletter
\newcommand{\nadaljevanje}{\dodatek{\newpage\noindent\emph{(\@sloeng{nadaljevanje rešitve \arabic{naloga}. naloge}{continuation of the answer to question \arabic{naloga}})}}}
\makeatother
\izpit
  [ucilnica=201,naloge=4]{Teorija programskih jezikov: 2. izpit}{1.\ julij 2022}{
}
\dodatek{
  \vspace{\stretch{1}}
  \begin{itemize}
    \item \textbf{Ne odpirajte} te pole, dokler ne dobite dovoljenja.
    \item Zgoraj \textbf{vpišite svoje podatke} in označite \textbf{sedež}.
    \item Na vidno mesto položite \textbf{dokument s sliko}.
    \item Preverite, da imate \textbf{telefon izklopljen} in spravljen.
    \item Čas pisanja je \textbf{180 minut}.
    \item Doseči je možno \textbf{70 točk}.
    \item Veliko uspeha!
  \end{itemize}
  \vspace{\stretch{3}}
  \newpage
}

%%%%%%%%%%%%%%%%%%%%%%%%%%%%%%%%%%%%%%%%%%%%%%%%%%%%%%%%%%%%%%%%%%%%%%%

\naloga[\tocke{15}]

V $\lambda$-računu definirajmo:
\[
  S = \fun{n} \fun{f} \fun{x} f (n \, f \, x)
\]

\podnaloga
Zapišite vse korake v evalvaciji izraza $S \, \big(\fun{f} \fun{x} f (f x)\big) \, (\fun{x} x + 5) \, 26$ v \emph{leni} semantiki malih korakov.

\podnaloga
S Hindley-Milnerjevim algoritmom izračunajte najbolj splošen tip izraza $S$.

\nadaljevanje

%%%%%%%%%%%%%%%%%%%%%%%%%%%%%%%%%%%%%%%%%%%%%%%%%%%%%%%%%%%%%%%%%%%%%%%

\naloga[\tocke{15}]

V $\lambda$-račun dodamo tip $\{ A \}$, ki predstavlja zamrznjene izraze tipa $A$, izraz $\mathtt{freeze} \, M$, ki predstavlja zamrznjen izraz $M$, ter izraz $\mathtt{thaw} \, M$, ki nadaljuje z izvajanjem poprej zamrznjenega izraza $M$.

Zapišite dodatna pravila za določanje tipov ter operacijsko semantiko malih korakov, tako da bo jezik še vedno zadoščal izreku o varnosti. Izreka o varnosti \emph{ni treba dokazovati}.

\nadaljevanje

%%%%%%%%%%%%%%%%%%%%%%%%%%%%%%%%%%%%%%%%%%%%%%%%%%%%%%%%%%%%%%%%%%%%%%%

\naloga[\tocke{20}]

Vzemimo $\lambda$-račun z običajno \emph{neučakano} semantiko:
%
\begin{align*}
  \text{tip}~A &\bnfis \boolty \mid A \to B \\
  \text{izraz}~M, N &\bnfis x \mid \true \mid \false \mid \cond{M}{N_1}{N_2} \mid \lambda x. M \mid M \, N \\
  \text{vrednost}~V &\bnfis \true \mid \false \mid \lambda x. M
  \intertext{
Operacijsko semantiko na krajši način lahko podamo tudi prek \emph{evalvacijskih kontekstov}}
  \text{kontekst}~E &\bnfis [\,] \mid \cond{E}{N_1}{N_2} \mid E \, N \mid V \, E
\end{align*}
%
Definiramo \emph{redukcijo} $M \rightarrowtail N$ s pravili:
%
\begin{mathpar}
  \infer{ }{\cond{\true}{N_1}{N_2} \rightarrowtail N_1} \and
  \infer{ }{\cond{\false}{N_1}{N_2} \rightarrowtail N_2} \and
  \infer{ }{(\lambda x. M) \, V \rightarrowtail M[V / x]}
\end{mathpar}
%
semantiko malih korakov $M \hookrightarrow N$ pa s pravilom:
\[
  \infer{M \rightarrowtail N}{E[M] \hookrightarrow E[N]}
\]
kjer je $E[M]$ izraz, ki ga dobimo, če v kontekstu $E$ luknjo $[\,]$ zamenjamo z izrazom $M$.

Dokažite, da $M \hookrightarrow N$ velja natanko takrat, kadar v običajni operacijski semantiki malih korakov velja $M \leadsto N$.

\nadaljevanje

%%%%%%%%%%%%%%%%%%%%%%%%%%%%%%%%%%%%%%%%%%%%%%%%%%%%%%%%%%%%%%%%%%%%%%%

\naloga[\tocke{20}]

Vzemimo varianto jezika \textsc{Imp}
\begin{align*}
  \text{aritmetični izraz } e &::=
    \ell \mid
    \intsym{n} \mid
    e_1 + e_2 \mid
    \cdots \\
  \text{logični izraz } b &::=
    \true \mid
    \false \mid
    e_1 = e_2 \mid
    \cdots \\
  \text{ukaz } c &::=
    \cond{b}{c_1}{c_2} \mid
    \whiledocmd{b}{c} \mid
    c_1 ; c_2 \mid
    \ell := e \mid
    \skipcmd
\end{align*}
kjer predpostavimo, da je množica lokacij $L$ vnaprej določena, pomnilnik pa je na vseh lokacijah vedno definiran. Branje $\ell$ tako vedno uspe, prirejanje $\ell := e$ pa vedno samo povozi obstoječe stanje.

Podajte denotacijsko semantiko naštetih izrazov in ukazov jezika \textsc{Imp}, ki naj bo netrivialna in zdrava, torej usklajena z operacijsko semantiko. Usklajenosti \emph{ni treba dokazovati}.

\nadaljevanje

\end{document}
