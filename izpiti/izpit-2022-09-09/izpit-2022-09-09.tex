\documentclass[arhiv]{../izpit}
\usepackage{amssymb}
\usepackage{fouriernc}
\usepackage{mathpartir}

\begin{document}

\newcommand{\bnfis}{\mathrel{{:}{:}{=}}}
\newcommand{\bnfor}{\;\mid\;}
\newcommand{\fun}[2]{\lambda #1. #2}
\newcommand{\conditional}[3]{\mathtt{if}\;#1\;\mathtt{then}\;#2\;\mathtt{else}\;#3}
\newcommand{\whileloop}[2]{\mathtt{while}\;#1\;\mathtt{do}\;#2}
\newcommand{\recfun}[3]{\mathtt{rec}\;#1\;#2. #3}
\newcommand{\funty}[2]{#1 \to #2}
\newcommand{\tru}{\mathtt{true}}
\newcommand{\fls}{\mathtt{false}}
\newcommand{\tbool}{\mathtt{bool}}
\newcommand{\intsym}[1]{\underline{#1}}
\newcommand{\kwd}[1]{\mathbf{#1}}
\newcommand{\kwdpre}[1]{\kwd{#1}\;}
\newcommand{\kwdmid}[1]{\;\kwd{#1}\;}
\newcommand{\kwdpost}[1]{\;\kwd{#1}}
\newcommand{\true}{\kwd{true}}
\newcommand{\false}{\kwd{false}}
\newcommand{\cond}[3]{\kwdpre{if} #1 \kwdmid{then} #2 \kwdmid{else} #3}
\newcommand{\whiledocmd}[2]{\kwdpre{while} #1 \kwdmid{do} #2}
\newcommand{\skipcmd}{\kwd{skip}}
\newcommand{\ttt}{\!t}
\newcommand{\fff}{\!\!f}
% \newcommand{\infer}[2]{\displaystyle{\frac{#1}{#2}}}
\newcommand{\boolty}{\kwd{bool}}
\newcommand{\intty}{\kwd{int}}
\newcommand{\eqs}{\mathcal{E}}
\newcommand{\ctxt}{\mathcal{C}}
\newcommand{\itp}[1]{[\![ #1 ]\!]}
\newcommand{\return}{\kwdpre{return}}
\newcommand{\letin}[1]{\kwdpre{let} #1 \kwdmid{in}}
\newcommand{\bind}{\mathop{>\!\!\!>\!\!\!=}}
\newcommand{\tand}{\mathbin{\mathtt{or}}}
\newcommand{\tandalso}{\mathbin{\mathtt{||}}}

\makeatletter
\newcommand{\nadaljevanje}{\dodatek{\newpage\noindent\emph{(\@sloeng{nadaljevanje rešitve \arabic{naloga}. naloge}{continuation of the answer to question \arabic{naloga}})}}}
\makeatother
\izpit
  [ucilnica=201,naloge=4]{Teorija programskih jezikov: 3. izpit}{9.\ september 2022}{
}
\dodatek{
  \vspace{\stretch{1}}
  \begin{itemize}
    \item \textbf{Ne odpirajte} te pole, dokler ne dobite dovoljenja.
    \item Zgoraj \textbf{vpišite svoje podatke} in označite \textbf{sedež}.
    \item Na vidno mesto položite \textbf{dokument s sliko}.
    \item Preverite, da imate \textbf{telefon izklopljen} in spravljen.
    \item Čas pisanja je \textbf{180 minut}.
    \item Doseči je možno \textbf{70 točk}.
    \item Veliko uspeha!
  \end{itemize}
  \vspace{\stretch{3}}
  \newpage
}

%%%%%%%%%%%%%%%%%%%%%%%%%%%%%%%%%%%%%%%%%%%%%%%%%%%%%%%%%%%%%%%%%%%%%%%

\naloga[\tocke{15}]

V $\lambda$-računu definirajmo:
\[
  M = (\fun{f} f \, f) \, (\fun{m} m + 6 * 7)
\]

\podnaloga
Zapišite vse korake v evalvaciji izraza $M$ v \emph{neučakani} semantiki malih korakov.

\podnaloga
Dokažite, da izraz $M$ nima veljavnega tipa.

\nadaljevanje

%%%%%%%%%%%%%%%%%%%%%%%%%%%%%%%%%%%%%%%%%%%%%%%%%%%%%%%%%%%%%%%%%%%%%%%

\naloga[\tocke{25}]
V $\lambda$-račun dodamo operaciji:
\[
  M \bnfis \cdots \bnfor
  M_1 \tand M_2 \bnfor
  M_1 \tandalso M_2
\]
Obe operaciji naj bi izračunali logično disjunkcijo Boolovih izrazov $M_1$ in $M_2$, razlika je le v tem, da $\tand$ evaluira~$M_2$ samo po potrebi, če iz~$M_1$ ni razviden rezultat, medtem ko $\tandalso$ vedno evaluira oba~$M_1$ in~$M_2$.

\podnaloga Zapišite pravila za operacijsko semantiko in določanje tipov za $\tand$ in $\tandalso$.\prostor[2]
\podnaloga Podajte primer izrazov~$M_1$ in~$M_2$ tipa $\tbool$,
  iz katerih je opazna razlika med $M_1 \tand M_2$ in $M_1 \tandalso M_2$.\prostor
\podnaloga Dokažite, da za razširjeni jezik še vedno velja izrek o varnosti.\prostor[2]

\nadaljevanje

%%%%%%%%%%%%%%%%%%%%%%%%%%%%%%%%%%%%%%%%%%%%%%%%%%%%%%%%%%%%%%%%%%%%%%%

\naloga[\tocke{20}]

Dokažite, da v $\lambda$-računu s preprostimi tipi za vsak sklep $\Gamma, \Gamma' \vdash M : A$, vsak tip $B$ in vsako spremenljivko~$x$, ki se ne pojavlja v $M$, velja tudi sklep $\Gamma, x : B, \Gamma' \vdash M : A$.

\nadaljevanje

%%%%%%%%%%%%%%%%%%%%%%%%%%%%%%%%%%%%%%%%%%%%%%%%%%%%%%%%%%%%%%%%%%%%%%%

\naloga[\tocke{20}]

Naj bo $\mathcal{D}_f(X)$ množica vseh tistih verjetnostnih porazdelitev na množici $X$, ki imajo končen nosilec.

\podnaloga Podajte ustrezni funkciji $\eta$ in $\bind$, za kateri bo $(\mathcal{D}_f, \eta, \bind)$ monada za interpretacijo učinka naključja. Dokažite, da monada zadošča zahtevanim zakonom.

\podnaloga Pokažite, kako z monado interpretirate razširitev drobnozrnatega $\lambda$-računa z izračunom $M_1 \oplus_p M_2$, ki z verjetnostjo $p$ izvede izračun $M_1$, z verjetnostjo $1 - p$ pa izračun $M_2$.



\nadaljevanje

\end{document}
